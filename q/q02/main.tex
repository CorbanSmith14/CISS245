\input{thispreamble.tex}

\renewcommand\AUTHOR{jdoe5@cougars.ccis.edu} % CHANGE TO YOURS

\begin{document}
\topmattertwo

If the code has an error (either it has a syntax error and
does not compile or it has a runtime error and crashes when you run it),
write ERROR.

For the first few quizzes, you enter your answer in \verb!main.txt!
and email me the file as an attachment.

This is a 7-minute, no-computer, closed-book quiz.
After you are done, you can check with a computer, look at my notes,
and make corrections.
But if you run out of time or have to make corrections, it means you have
not fully studied my CISS240 materials.

%------------------------------------------------------------------------------
\nextq
The output of the following code fragment is \answerbox{}
\begin{console}
std::cout << 100 % 3;
\end{console}

%------------------------------------------------------------------------------
\nextq
The output of the following code fragment is \answerbox{}
\begin{console}
std::cout << 100 % 2;
\end{console}

%------------------------------------------------------------------------------
\nextq
The output of the following code fragment is \answerbox{}
\begin{console}
std::cout << 100 % 1;
\end{console}

%------------------------------------------------------------------------------
\nextq
The output of the following code fragment is \answerbox{}
\begin{console}
std::cout << 100 % 0;
\end{console}

%------------------------------------------------------------------------------
\nextq
The output of the following code fragment is \answerbox{}
\begin{console}
std::cout << 100 % 100;
\end{console}

%------------------------------------------------------------------------------
\nextq
Write a function that computes the sum of the
values in array \verb!x! from index 0 up to
and including index \verb!size - 1!.
The function prototype is
\begin{console}
double sum(double x[], int size);
\end{console}
\textsc{Answer:}\vspace{-2mm}
\begin{answercode}

\end{answercode}

%------------------------------------------------------------------------------
\nextq
Write a function that
computes the maximum of the
values in array \verb!x! from index 0 up to
and including index \verb!size - 1!.
The function prototype is
\begin{console}
int max(int x[], int size);
\end{console}
\textsc{Answer:}\vspace{-2mm}
\begin{answercode}

\end{answercode}

%------------------------------------------------------------------------------
\nextq
Write a function that performs a linear search
for \verb!target! in array \verb!x! from index 0 up to
and including index \verb!size - 1!.
If \verb!target! is not found, \verb!-1! is returned.
The function prototype is
\begin{console}
int linearsearch(int x[], int size, int target);
\end{console}   
\textsc{Answer:}\vspace{-2mm}
\begin{answercode}

\end{answercode}

%------------------------------------------------------------------------------
\nextq
Write a function that counts the number of times
\verb!target! appears in 
array \verb!x! from index 0 up to
and including index \verb!size - 1!.
The function prototype is
\begin{console}
int count(int x[], int size, int target);
\end{console}   
\textsc{Answer:}\vspace{-2mm}
\begin{answercode}

\end{answercode}

%------------------------------------------------------------------------------
\nextq
Write a function that swaps the values of two integer variables.
Part of the function prototype is given:
\begin{console}
    swap(     x,    y);
\end{console}   
\textsc{Answer:}\vspace{-2mm}
\begin{answercode}

\end{answercode}

%------------------------------------------------------------------------------
\newpage
\input{instructions.tex}

\end{document}
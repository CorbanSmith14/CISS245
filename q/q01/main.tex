\input{thispreamble.tex}

\renewcommand\AUTHOR{crsmith14@cougars.ccis.edu} % CHANGE TO YOURS

\begin{document}
\topmattertwo

If the code has an error (either it has a syntax error and
does not compile or it has a runtime error and crashes when you run it),
write ERROR.

For the first few quizzes, you enter your answer in \verb!main.txt!
and email me the file as an attachment.

This is a 10-minute, no computer quiz. After you are done, you can
check with a computer ane make corrections.
But if you run out of time or have to make corrections, it means you have
not fully studied my CISS240 materials.

Q0. $41 + 1 = $ \answerbox{41}

\nextq
The string length of
\verb!"hello \tworld\n???\n"!
is \answerbox{}.


\nextq
In the following code fragment, the output is \verb!132!.
The value of integer variable \verb!n! is a power of 10.
Therefore the value of \verb!n! is \answerbox{}.
\begin{Verbatim}[frame=single]
std::cout << 132435 / n << '\n';
\end{Verbatim}


\nextq
In the following code fragment, the output is \verb!2435!.
The value of integer variable \verb!n! is a power of 10.
Therefore the value of \verb!n! is \answerbox{}.
\begin{Verbatim}[frame=single]
std::cout << 132435 % n << '\n';
\end{Verbatim}


\nextq
In the following code fragment, the output is \verb!32!.
The values of integer variables \verb!m! and \verb!n! are
powers of 10.
Therefore
the value of \verb!m! is \answerbox{}
and
the value of \verb!n! is \answerbox{}
\begin{Verbatim}[frame=single]
std::cout << 132435 / m % n << '\n';
\end{Verbatim}


\nextq
\tf:
(T = true, F = false, M = statement is meaningless and cannot be answered.)
You cannot assign an integer value to a \verb!double! variable.
In other words the code fragment below is invalid C++.
\dotfill\answerbox{}
\begin{Verbatim}[frame=single]
double x = 42;
\end{Verbatim}

\nextq
The output of the following code fragment is 
\dotfill\answerbox{}
\begin{Verbatim}[frame=single]
int x = 4, y = 0;
std::cout << x / y;
\end{Verbatim}

\nextq
The output of the following code fragment is 
\dotfill\answerbox{}
\begin{Verbatim}[frame=single]
int x = 4, y = 1;
std::cout << x / y;
\end{Verbatim}

\nextq
The output of the following code fragment is 
\dotfill\answerbox{}
\begin{Verbatim}[frame=single]
int x = 4, y = 4;
std::cout << x / y;
\end{Verbatim}

\nextq
The output of the following code fragment is 
\dotfill\answerbox{}
\begin{Verbatim}[frame=single]
int x = 4, y = 8;
std::cout << x / y;
\end{Verbatim}

\nextq
The output of the following code fragment is 
\dotfill\answerbox{}
\begin{Verbatim}[frame=single]
int x = 4, y = 8;
std::cout << double(x / y);
\end{Verbatim}



\newpage
\input{instructions.tex}

\end{document}
\begin{Verbatim}[frame=single]
